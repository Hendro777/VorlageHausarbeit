% !TEX root = ../Projektdokumentation.tex

% Vorgaben IHK -----------------------------------------------------------------
% Schriftgröße: 11
% Zeilenabstand: 1,5 zeilig
% Rand: 2,5 cm (links, rechts, oben, unten)
% Ausrichtung: Blocksatz --> Standardmäßig in LaTeX aktiv
% Seitenzahlen: Fußzeile, zentriert, beginnend mit der Dokumentation
% Maximale Tiefe: 1.1.1 / subsubsections

\geometry{a4paper, margin=25mm}

\usepackage[
	automark, % Kapitelangaben in Kopfzeile automatisch erstellen
	headsepline, % Trennlinie unter Kopfzeile
	ilines % Trennlinie linksbündig ausrichten
]{scrlayer-scrpage}

\pagestyle{scrheadings}

% Kopfzeile ---------------------------------------------------------------------
\renewcommand*\sectionmarkformat{} % Keine Nummerierung im Kopf
\renewcommand{\headfont}{\normalfont} % Schriftform der Kopfzeile
\ihead{\large{\titel}\\[1ex] \textit{\headmark}}
\automark{section}
\chead{}
\ohead{\includegraphics[scale=0.2]{\betriebLogo}}
\setlength{\headheight}{15mm} % Höhe der Kopfzeile

% Fußzeile ---------------------------------------------------------------------
\ifoot{\autorName}
\cfoot{\pagemark}
\ofoot{\betriebName}
\setlength{\footheight}{10mm} % Höhe der Fußzeile

% Abstand zwischen Nummerierung und Überschrift definieren
\newcommand{\headingSpace}{1.0 cm}

% Abschnittsüberschriften im selben Stil wie beim Inhaltsverzeichnis einrücken
\renewcommand*{\othersectionlevelsformat}[3]{
  \makebox[\headingSpace][l]{#3\autodot}
}

%\cftsetindents{chapter}{0.0cm}{\headingSpace}
\cftsetindents{section}{0.0cm}{\headingSpace}
\cftsetindents{subsection}{1.0cm}{1.0 cm}
\cftsetindents{subsubsection}{2.0cm}{1.45 cm}
\cftsetindents{figure}{0.0cm}{\headingSpace}
\cftsetindents{table}{0.0cm}{\headingSpace}


% Allgemeines -----------------------------------------------------------------

\setlength{\textheight}{665pt} % Größe des "Textfeldes"

\onehalfspacing % Zeilenabstand 1,5 Zeilen
\frenchspacing % erzeugt ein wenig mehr Platz hinter einem Punkt

\usepackage{blindtext} % Lipsum

\counterwithout{footnote}{section} % Fußnoten fortlaufend durchnummerieren
\setcounter{tocdepth}{3} % im Inhaltsverzeichnis werden die Kapitel bis zum Level der subsubsection übernommen
\setcounter{secnumdepth}{3} % Kapitel bis zum Level der subsubsection werden nummeriert

% Aufzählungen anpassen
\renewcommand{\labelenumi}{\arabic{enumi}.}
\renewcommand{\labelenumii}{\arabic{enumi}.\arabic{enumii}.}
\renewcommand{\labelenumiii}{\arabic{enumi}.\arabic{enumii}.\arabic{enumiii}}

% Tabellenfärbung
\definecolor{heading}{rgb}{0.64,0.78,0.86}
\definecolor{odd}{rgb}{0.9,0.9,0.9}

\usepackage{float}

